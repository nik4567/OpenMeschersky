Модуль силы всемирного тяготения,действующий на материальную точку массы $m$,определяется равенством $F=m/mu/r^2$ ,где $/mu=fM$-гравитаци
онный параметр притягивающего центра ($M$-его масса,$f$-гравитационная постоянная) и $r$-расстояние от центра притяжения до притягиваемой точки.
Зная радиус $R$ небесного тела и ускорение $g$ силы тяжести) на его поверхности,определить гравитационный параметр $/mu$ небесного тела и вычис
лить его для Земли,если её радиус $R=6370$ км,а $g=9,81$ м/с$^2$.
