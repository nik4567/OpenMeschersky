Состояния равновесия в пространстве ($\theta$, $\Omega$, $\omega$) образуют
поверхность $\Pi$, уравнение которой
$(C + ma^2)\Omega\omega - A\Omega ^2\sin{\theta} + mga\sin{\theta} = 0$,
представляющую двумерное многообразие стационарных движений диска.
На этой поверхности точки прямой $\theta = \Omega = 0$ соответствуют
такому качению диска по прямой, при котором плоскость диска сохраняет
вертикальное положение.
Точки прямой $\theta = \omega = 0$ соответствуют верчению диска вокруг
неподвижного вертикального диаметра.
Все остальные точки поверхности $\Pi$ соответствуют круговым движениям.
