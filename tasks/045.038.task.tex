Многоступенчатая ракета состоит из полезного груза и ступеней.
Каждая ступень после израсходования топлива отделяется от остальной конструкции.
Под субракетой понимается сочетание работающей ступени,
всех неработающих ступеней и полезного груза,
причём для данной субракеты все неработающие ступени и полезный груз
являются <<полезным грузом>> , т. е. каждая ракета
рассматривается как одноступенчатая ракета.
На рисунке указана нумерация ступеней и субракеты.

Пусть $q$ --- вес полезного груза, $P_i$ --- вес топлива в $i$-й ступени,
$Q_i$ --- сухой (без топлива) вес $i$-й ступени,
$G_i$ --- полный вес $i$-й субракеты.

Вводя в рассмотрение число Циолковского для каждой субракеты
$$z_i = \frac{G_i}{G_i - P_i}$$
и конструктивную характеристику
(отношение полного веса ступени к её сухому весу)
для каждой ступени
$$s_i = \frac{Q_i + P_i}{Q_i},$$
определить:
\begin{enumerate}
\item полный стартовый вес всей ракеты,
\item вес $k$-й субракеты,
\item вес топлива $k$-й ступени,
\item сухой вес $k$-й ступени.
\end{enumerate}
