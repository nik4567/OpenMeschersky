При наезде тележкт $A$ на упругий упор $B$ начинаются колебания
подвешенного на стержне груза $D$.
Составить дифференциальные уравнения движения материальной системы,
если $m_1$ --- масса тележки, $m_2$ --- масса груза, $l$ --- длина стержня,
$c$ --- коэффициент жесткости пружины упора $B$.
Массой колес и всеми силами сопротивления пренебречь.
Начало отсчета оси $x$ взять в левом конце недеформированной пружины.
Определить период малых колебаний груза при отсутствии упора $B$.
Массой стержня пренебречь.
