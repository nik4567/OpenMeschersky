На одном и том же основании, совершающем горизонтальные случайные колебания
по одной оси, горизонтально установлены три линейных акселерометра,
имеющих одинаковые статические характеристики,
но различные динамические свойства.
Первый из них имеет собственную частоту $\omega _0$ и относительную высоту
резонансного пика, равную $1.2$, второй --- ту же собственную частоту,
но относительную высоту резонансного пика, равную $1.6$,
третий --- собственную частоту $2\omega _0$, а относительную высоту
резонансного пика, как у первого акселерометра.
Предполагая, что случайное ускорение при колебаниях основания
можно считать белым шумом, определить, насколько различаются средние
квадратические значения $\sigma _1$, $\sigma _2$ и $\sigma _3$
выходных сигналов этих акселерометров.
