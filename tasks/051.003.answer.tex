\begin{enumerate}
	\item $v_1 = \sqrt{\frac{\mu}{r}} =
	\sqrt{\frac{gR^2}{R + H}}$
	(круговая скорость на высоте $H$ для данного небесного тела);
	\item $T = 2\pi r\sqrt{\frac{r}{mu}} =
	2\pi\frac{(R+H)^{ 3/2 }}{R\sqrt{g}}$.
	Здесь $r$ --- расстояние от материальной точки до центра небесного тела,
	$\mu$ --- его гравитационный параметр,
	$g$ ---	ускорение силы тяжести на его поверхности.
\end{enumerate}
