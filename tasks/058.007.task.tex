Однородная прямоугольная платформа массы $1000$ кг подвешена к опоре
на четырёх тросах одинаковой длины, сходящихся в одной точке.
Расстояние платформы до точки подвеса равно $h = 2$ м.
На платформу установлены четыре груза малых размеров.
Массы и расположение грузов случайны.
Предполагается, что массы грузов и их прямоугольные координаты
$x_i$ и $y_i$, отсчитываемые от центра платформы, взаимно независимы
и имеют гауссовское распределение.
Математические ожидания масс всех четырёх грузов одинаковы
и равны $m_M = 100$ кг, среднеквадратические отклонения также одинаковы
и равны $\sigma _M = 20$ кг.
Координаты грузов имеют нулевые математические ожидания,
средние квадратические отклонения координат равны $\sigma _x = 0.5$ м
и $\sigma _y = 0.7$ м.
Определить границы таких симметричных интервалов для углов наклона
$\theta _x$ и $\theta _y$ платформы, находящейся в равновесии
при установленных грузах, вероятности нахождения в которых равны $0.99$.
Углы считать малыми.
