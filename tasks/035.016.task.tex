На средней скамейке лодки, находившейся в покое, сидели два человека.
Один из них, массы $M_1 = 50$ кг, переместился вправо на нос лодки.
В каком направлении и на какое расстояние должен переместиться
второй человек массы $M_2 = 70$ кг для того,
чтобы лодка осталась в покое?
Длинна лодки $4$ м. Сопротивлением воды движению лодки пренебречь.
